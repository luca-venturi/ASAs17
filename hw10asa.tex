\documentclass[a4paper,11pt]{article}

\usepackage{geometry}
\usepackage{listings}
\usepackage{mathrsfs}
\usepackage[font={small},labelfont={sf,bf}]{caption}
\usepackage{color}
\usepackage{amssymb}
\usepackage[utf8]{inputenc}
\usepackage{afterpage}
\usepackage[T1]{fontenc}
\usepackage{amsfonts}
\usepackage{mathtools}
\usepackage{amsmath}
\usepackage{verbatim}
\usepackage{bm}
\usepackage{bbm}
\usepackage{enumerate}
\usepackage{amsthm}
\usepackage{stmaryrd}

\geometry{a4paper,top=3cm,bottom=3cm,left=2cm,right=2cm,heightrounded, bindingoffset=5mm}

\theoremstyle{definition}
\newtheorem{definition}{Definition}[section]
\theoremstyle{plain}
\newtheorem{theo}[definition]{Theorem}
\newtheorem{prop}[definition]{Proposition}
\newtheorem{lemma}[definition]{Lemma}
\newtheorem{cor}[definition]{Corollary}
\newtheorem{ex}[definition]{Example}
\theoremstyle{remark}
\newtheorem{rem}[definition]{Remark}
\newtheorem{rem*}[definition]{}

\newcommand*\mcup{\mathbin{\mathpalette\mcupinn\relax}}
\newcommand*\mcupinn[2]{\vcenter{\hbox{$\mathsurround=0pt
  \ifx\displaystyle#1\textstyle\else#1\fi\bigcup$}}}
\newcommand*\mcap{\mathbin{\mathpalette\mcapinn\relax}}
\newcommand*\mcapinn[2]{\vcenter{\hbox{$\mathsurround=0pt
  \ifx\displaystyle#1\textstyle\else#1\fi\bigcap$}}}
\DeclarePairedDelimiter{\abs}{\lvert}{\rvert}
\DeclarePairedDelimiter{\norm}{\lVert}{\rVert}
\DeclarePairedDelimiter{\parr}{(}{)}
\DeclarePairedDelimiter{\parq}{[}{]}
\DeclarePairedDelimiter{\parqq}{\llbracket}{\rrbracket}
\DeclarePairedDelimiter{\bra}{\lbrace}{\rbrace}
\DeclarePairedDelimiter{\ceil}{\lceil}{\rceil}
\DeclarePairedDelimiter{\prodscal}{\langle}{\rangle}
\DeclarePairedDelimiter{\floor}{\lfloor}{\rfloor}
\DeclareMathOperator*{\argmin}{arg\,min}
\DeclareMathOperator*{\argmax}{arg\,max}
\DeclareMathOperator*{\expval}{\mathbb{E}}
\DeclareMathOperator*{\varval}{\mathrm{Var}}
\DeclareMathOperator*{\covval}{\mathrm{Cov}}

\begin{document}

\title{Applied Stochastic Analysis \\ Homework assignment 10}
\author{Luca Venturi}
\maketitle

\section*{Exercise 1}

\paragraph*{(a)}

By applying It\^o formula, we have
\begin{align*}
I^{(3)}_t & \doteq \int_0^t\int_0^{s_1}\int_0^{s_2} dW_{s_3}\,dW_{s_2}\,dW_{s_1} = \int_0^t\int_0^{s_1}W_{s_2}\,dW_{s_2}\,dW_{s_1} \\
& = \int_0^t\parq*{\int_0^{s_1}\,d\parr*{\frac{1}{2}W_{s_2}^2} - \frac{1}{2}\int_0^{s_1}ds_2}\,dW_{s_1} = \frac{1}{2}\int_0^t W_{s_1}^2\,dW_{s1} - \frac{1}{2}\int_0^t s_1\,dW_{s1} \\ & = \frac{1}{6}\int_0^td\parr*{W_{s_1}^3} - \frac{1}{2}\int_0^tW_{s_1}\,ds_1 - \frac{1}{2}\int_0^td(s_1W_{s_1}) + \frac{1}{2}\int_0^t W_{s_1}\,ds_1 \\ & = \frac{1}{6}W_t^3 -\frac{1}{2}tW_t .
\end{align*}

\paragraph*{(b)}

We proceed by induction. For $k=3$ the equality holds due to part (a) and to 
$$
I^{(2)}_t = \frac{1}{2}W_t^2-\frac{1}{2}t,\qquad I^{(1)}_t = W_t.
$$
First of all, we notice that
$$
I_t^{(k+1)} = \int_0^tI_{s_1}^{(k)}\,dW_{s_1} \qquad\Longleftrightarrow\qquad dI_t^{(k+1)} = I_t^{(k)}\,dW_t.
$$
Then, if we suppose that the recursion formula holds for $k$, for $k+1$ we get
\begin{align*}
dI_t^{(k+1)} & = I_t^{(k)}\,dW_t = \frac{1}{k}\parr*{W_tI_t^{(k-1)}-tI_t^{(k-2)}}\,dW_t. 
\end{align*}
On the other hand
\begin{align*}
d\parq*{\frac{1}{k+1}\parr*{W_tI_t^{(k)}-tI_t^{(k-2)}}} & = \frac{1}{k+1}\parq*{d\parr*{W_tI_t^{(k)}}-d\parr*{tI_t^{(k-1)}}} \\ & = \frac{1}{k+1}\parq*{I_t^{(k)}dW_t+W_t\,dI_t^{(k)}+dW_t\,dI_t^{(k)}-tdI_t^{(k-1)}-I_t^{(k-1)}\,dt -dtdI_t^{(k-1)}} \\ & = 
\end{align*}


\end{document}