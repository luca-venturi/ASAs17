\documentclass[a4paper,11pt]{article}

\usepackage{geometry}
\usepackage{listings}
\usepackage{mathrsfs}
\usepackage[font={small},labelfont={sf,bf}]{caption}
\usepackage{color}
\usepackage{amssymb}
\usepackage[utf8]{inputenc}
\usepackage{afterpage}
\usepackage[T1]{fontenc}
\usepackage{amsfonts}
\usepackage{mathtools}
\usepackage{amsmath}
\usepackage{verbatim}
\usepackage{bm}
\usepackage{bbm}
\usepackage{enumerate}
\usepackage{amsthm}
\usepackage{stmaryrd}

\geometry{a4paper,top=3cm,bottom=3cm,left=2cm,right=2cm,heightrounded, bindingoffset=5mm}

\theoremstyle{definition}
\newtheorem{definition}{Definition}[section]
\theoremstyle{plain}
\newtheorem{theo}[definition]{Theorem}
\newtheorem{prop}[definition]{Proposition}
\newtheorem{lemma}[definition]{Lemma}
\newtheorem{cor}[definition]{Corollary}
\newtheorem{ex}[definition]{Example}
\theoremstyle{remark}
\newtheorem{rem}[definition]{Remark}
\newtheorem{rem*}[definition]{}

\newcommand*\mcup{\mathbin{\mathpalette\mcupinn\relax}}
\newcommand*\mcupinn[2]{\vcenter{\hbox{$\mathsurround=0pt
  \ifx\displaystyle#1\textstyle\else#1\fi\bigcup$}}}
\newcommand*\mcap{\mathbin{\mathpalette\mcapinn\relax}}
\newcommand*\mcapinn[2]{\vcenter{\hbox{$\mathsurround=0pt
  \ifx\displaystyle#1\textstyle\else#1\fi\bigcap$}}}
\DeclarePairedDelimiter{\abs}{\lvert}{\rvert}
\DeclarePairedDelimiter{\norm}{\lVert}{\rVert}
\DeclarePairedDelimiter{\parr}{(}{)}
\DeclarePairedDelimiter{\parq}{[}{]}
\DeclarePairedDelimiter{\parqq}{\llbracket}{\rrbracket}
\DeclarePairedDelimiter{\bra}{\lbrace}{\rbrace}
\DeclarePairedDelimiter{\ceil}{\lceil}{\rceil}
\DeclarePairedDelimiter{\prodscal}{\langle}{\rangle}
\DeclarePairedDelimiter{\floor}{\lfloor}{\rfloor}
\DeclareMathOperator*{\argmin}{arg\,min}
\DeclareMathOperator*{\argmax}{arg\,max}
\DeclareMathOperator*{\expval}{\mathbb{E}}
\DeclareMathOperator*{\varval}{\mathrm{Var}}
\DeclareMathOperator*{\covval}{\mathrm{Cov}}

\begin{document}

\title{Applied Stochastic Analysis \\ Homework assignment 9}
\author{Luca Venturi}
\maketitle

\section*{Exercise 1}

Using the formula to transform an It\^o SDE into a Stratonovich one (or viceversa) we get:

\paragraph*{(a)}

$$
dX_t = \parr*{aX_t + \frac{1}{2}b^2X_t} \,dt + bX_t\,dB_t. 
$$

\paragraph*{(b)}

$$
dX_t = \frac{1}{2}\parr*{\sin X_t\cos X_t - t^2\sin X_t} \,dt + (t^2 + \cos X_t)\,dB_t. 
$$

\paragraph*{(c)}

$$
dX_t = \parr*{r-\frac{1}{2}\alpha^2}X_t\,dt + \alpha X_t\circ dB_t. 
$$

\paragraph*{(d)}

$$
dX_t = \parr*{2e^{-X_t} - X_t^3}\,dt + X_t^2\circ dB_t.
$$

\section*{Exercise 2}

\paragraph*{(a)}

By It\^o formula we have 
$$
d(X^n_t) = \parr*{\lambda n + \frac{1}{2}n(n-1)\sigma^2}X^n_t\,dt + X^n_t\,dW_t,
$$
i.e.
$$
X_t^n = x_0^n + \int_0^t \parr*{\lambda n  + \frac{1}{2}n(n-1)\sigma^2}X^n_s\,ds + \sigma n \int_0^t X^n_s\,dW_s.
$$
Taking the expectation we get
$$
M_n(t) = \expval X_t^n = \expval x_0^n + \int_0^t \parr*{\lambda n  + \frac{1}{2}n(n-1)\sigma^2}M_n(s)\,ds.
$$
The above formula is equivalent to say that $M_n$ satisfies the equation
\begin{equation}\label{ex2:ode}
\frac{dM_n}{dt} = \parr*{\lambda n +\frac{\sigma^2}{2}n(n-1)}M_n, \qquad M_n(0) = \expval x_0^n.
\end{equation}

\paragraph*{(b)}

The solution to equation (\ref{ex2:ode}) is 
$$
M_n(t) = \parr*{\expval x_0^n}\,e^{n(\lambda + \sigma^2(n-1)/2)t}.
$$

\paragraph*{(c)}

It must be 
$$
\lambda + \frac{\sigma^2}{2}(n-1) < 0, \qquad \text{i.e.} \qquad \lambda < - \frac{\sigma^2}{2}(n-1).
$$

\paragraph*{(d)}

One just need to take $N = \inf\bra{n\geq1\,:\,\lambda+\sigma^2(n-1)/2>0}$.

\section*{Exercise 3}

\paragraph*{(a)}

Suppose that $X_t$ satisfies the SDE 
\begin{equation}
\begin{cases} dX_t = a(X_t)\,dY_t \\ dY_t = b(X_t)\,dW_t \end{cases}
\label{ex3a:sys}
\end{equation}
The terms in the Riemann sums for (\ref{ex3a:sys}) are
$$
\begin{cases} \Delta_j X  = a(X_j)\Delta_j Y \\  \Delta_j Y = b(X_j)\Delta_j W \end{cases} \quad\Longrightarrow\quad \Delta_j X = a(X_j)b(X_j)\Delta_j W.
$$
So the above gives 
$$
dX_t = (a(X_t)b(X_t))\,dW_t.
$$

\paragraph*{(b)}

Suppose that $X_t$ satisfies the SDE 
\begin{equation}
\begin{cases} dX_t = a(X_t)\,dY_t \\ dY_t = b(X_t)\circ dW_t \end{cases}
\label{ex3b:sys}
\end{equation}
and that this implies
\begin{equation} 
\label{ex3b:ito}
dX_t = \alpha(X_t)\,dt + \beta(X_t)\,dW_t
\end{equation}
for some functions $\alpha$, $\beta$.
The terms in the Riemann sums for (\ref{ex3b:sys}) are
\begin{equation}
\begin{cases} \Delta_j X  = a(X_j)\Delta_j Y \\  \Delta_j Y = \frac{1}{2}(b(X_j)+b(X_{j+1}))\Delta_j W \end{cases} \label{ex3b:dis_sys}
\end{equation}
Now, using (\ref{ex3b:ito}), we have
$$
b(X_{j+1}) = b(X_j) + b'(X_j)\Delta_j X \quad\Longrightarrow\quad b(X_{j+1})\Delta_jW = b(X_j)\Delta_jW + b'(X_j)\beta(X_j)\Delta_jt,
$$
and thus from (\ref{ex3b:dis_sys}) we get
$$
\Delta_j X = \frac{1}{2}a(X_j)b'(X_j)\beta(X_j)\Delta_j t + a(X_j)b(X_j)\Delta_jW 
$$
Comparing the above with (\ref{ex3b:ito}), we get that $X$ must satisfy:
$$
dX_t = \frac{1}{2}a^2(X_t)b(X_t)b'(X_t)\,dt + a(X_t)b(X_t)\,dW_t.
$$

\paragraph*{(c)}

Suppose that $X_t$ satisfies the SDE 
\begin{equation}
\begin{cases} dX_t = a(X_t)\circ dY_t \\ dY_t = b(X_t)\,dW_t \end{cases}
\label{ex3c:sys}
\end{equation}
and that this implies
\begin{equation} 
\label{ex3c:ito}
dX_t = \alpha(X_t)\,dt + \beta(X_t)\,dW_t
\end{equation}
for some functions $\alpha$, $\beta$.
The terms in the Riemann sums for (\ref{ex3c:sys}) are
\begin{equation}
\begin{cases} \Delta_j X  = \frac{1}{2}(a(X_j)+a(X_{j+1}))\Delta_j Y \\  \Delta_j Y = b(X_j)\Delta_j W \end{cases} \label{ex3c:dis_sys}
\end{equation}
Now, using (\ref{ex3c:ito}), we have
$$
a(X_{j+1}) = a(X_j) + a'(X_j)\Delta_j X \quad\Longrightarrow\quad a(X_{j+1})\Delta_jW = a(X_j)\Delta_jW + a'(X_j)\beta(X_j)\Delta_jt,
$$
and thus from (\ref{ex3c:dis_sys}) we get
$$
\Delta_j X = \frac{1}{2}a'(X_j)b(X_j)\beta(X_j)\Delta_j t + a(X_j)b(X_j)\Delta_jW 
$$
Comparing the above with (\ref{ex3c:ito}), we get that $X$ must satisfy:
$$
dX_t = \frac{1}{2}a(X_t)a'(X_t)b^2(X_t)\,dt + a(X_t)b(X_t)\,dW_t.
$$

\paragraph*{(d)}

Suppose that $X_t$ satisfies the SDE 
\begin{equation}
\begin{cases} dX_t = a(X_t)\circ dY_t \\ dY_t = b(X_t)\circ dW_t \end{cases}
\label{ex3d:sys}
\end{equation}
and that this implies
\begin{equation} 
\label{ex3d:ito}
dX_t = \alpha(X_t)\,dt + \beta(X_t)\,dW_t
\end{equation}
for some functions $\alpha$, $\beta$.
The terms in the Riemann sums for (\ref{ex3c:sys}) are
\begin{equation}
\begin{cases} \Delta_j X  = \frac{1}{2}(a(X_j)+a(X_{j+1}))\Delta_j Y \\  \Delta_j Y = \frac{1}{2}(b(X_j)+b(X_{j+1}))\Delta_j W \end{cases} \label{ex3d:dis_sys}
\end{equation}
Now, using (\ref{ex3d:ito}), we have
$$
a(X_{j+1}) = a(X_j) + a'(X_j)\Delta_j X, \qquad b(X_{j+1}) = b(X_j) + b'(X_j)\Delta_j X ,
$$
and thus from (\ref{ex3d:dis_sys}) we get
\begin{align*}
\Delta_j X & = \frac{1}{4}(a(X_j)+a(X_{j+1}))(b(X_j)+b(X_{j+1}))\Delta_jW \\ & = \frac{1}{4}(2a(X_j)+a'(X_j)\beta(X_j)\Delta_jW)(2b(X_j)+b'(X_j)\beta(X_j)\Delta_jW)\Delta_jW \\ & = \frac{1}{2}(a\cdot b)'(X_j)\beta(X_j)\Delta_jt + a(X_j)b(X_j)\Delta_jW
\end{align*}
Comparing the above with (\ref{ex3d:ito}), we get that $X$ must satisfy:
$$
dX_t = \frac{1}{2}(a\cdot b)(X_t)(a\cdot b)'(X_t)\,dt + a(X_t)b(X_t)\,dW_t.
$$

\section*{Exercise 4}

Be $Y_t = f(X_t)$ and suppose that $dX_t = \alpha(X_t)\,dt + \beta(X_t)\circ dW_t$. Then
$$
dX_t = (\alpha(X_t)+\frac{1}{2}\beta(X_t)\beta'(X_t))dt + \beta(X_t)\,dW_t
$$
and so, by It\^o formula
\begin{align}
dY_t & = f'(X_t)\,dX_t + \frac{1}{2}f''(X_t)(dX_t)^2 \notag\\ & = \parq*{\alpha(X_t)f'(X_t) +\frac{1}{2}\beta(X_t)\beta'(X_t)f'(X_t) + \frac{1}{2}\beta(X_t)^2f''(X_t)}dt + \beta(X_t)f'(X_t)\,dW_t \notag\\ & = \parq*{\alpha(X_t)f'(X_t) +\frac{1}{2}\beta(X_t)(\beta\cdot f')'(X_t)}dt + \beta(X_t)f'(X_t)\,dW_t. \label{ex:4}
\end{align}
If $g = f^{-1}$, $h = f'$, $\hat{\alpha} = \alpha\circ g$, $\hat{\beta} = \beta\circ g$ and $\hat{h} = h\circ g$,  then using the change of variable $y=f(x)$ (which we can do locally) (\ref{ex:4}) becomes
\begin{align*}
dY_t & = \parq*{\hat{\alpha}(Y_t)\hat{h}(Y_t)+\frac{1}{2}\hat{\beta}(Y_t)\hat{h}(Y_t)\frac{d}{dy}(\hat{\beta}\cdot\hat{h})(Y_t)}dt + \hat{\beta}(Y_t)\hat{h}(Y_t)\,dW_t \\ & = \hat{\alpha}(Y_t)\hat{h}(Y_t)\,dt + \hat{\beta}(Y_t)\hat{h}(Y_t)\circ dW_t \\ & = \hat{h}(Y_t)\parq*{\hat{\alpha}(Y_t)\,dt + \hat{\beta}(Y_t)\circ dW_t} \\ & = h(X_t)\parq*{\alpha(X_t)\,dt + \beta(X_t)\circ dW_t} = h(X_t) \circ  dX_t,
\end{align*}
i.e. $d(f(X_t)) = f'(X_t)\circ dX_t$.

\section*{Exercise 5}

It is sufficient to show that $dY_t = 0$, where $Y_t \doteq X_{1,t}^2+X_{2,t}^2$. Indeed, using the result from Exercise 4, it holds:
\begin{align*}
dY_t &  = d(X_{1,t}^2) + d(X_{2,t}^2) = 2X_{1,t}\circ dX_{1,t} + 2X_{2,t}\circ dX_{2,t} \\ & = \frac{1}{\abs{X_t}^2}\parq*{2X_{1,t}(X_{2,t}^2\circ dW_{1,t}-X_{1,t}X_{2,t}\circ dW_{2,t}) + 2X_{2,t}(-X_{1,t}X_{2,t}\circ dW_{1,t} + X_{1,t}^2\circ dW_{2,t})} \\ & = \frac{1}{\abs{X_t}^2}\parq*{(2X_{1,t}X_{2,t}^2 - 2X_{1,t}X_{2,t}^2)\circ dW_{1,t} + (2X_{1,t}^2X_{2,t} - 2X_{1,t}^2X_{2,t})\circ dW_{2,t}} = 0.
\end{align*}

\section*{Exercise 6}

\paragraph*{(a)}

The spectral form of the SDE is
$$
i\xi \hat{X}_\xi = -\alpha\hat{X}_\xi + \sigma\hat{\eta}_\xi \quad\Longrightarrow\quad \hat{X}_\xi = \frac{\sigma}{\alpha+i\xi}\cdot\hat{\eta}_\xi
$$
where $\hat{X}_\xi = dZ_X(\xi)$, $\hat{\eta}_\xi = dZ_\eta(\xi)$ ($\eta$ is a Gaussian process with mean $0$ and covariance function $C_\eta(t,s)=\delta(t-s)$).
Since the covariance function of $\eta$ has the spectral representation
$$
C_\eta(t) = \delta(t) = \int e^{i\xi t}\frac{d\xi}{2\pi}
$$
then the covariance function of $X$ has the spectral representation
$$
C_X(t) = \expval\parq{X_t\overline{X}_0} = \int e^{i\xi t}\expval\parq*{\hat{X}_\xi\overline{\hat{X}}_\lambda} = \int e^{i\xi t}\frac{\sigma}{(\alpha+i\xi)}\frac{\sigma}{(\alpha-i\lambda)}\expval[\hat{\eta}_\xi\overline{\hat{\eta}}_\lambda] = \int e^{i\xi t}\frac{\sigma^2}{\alpha^2+\xi^2}\frac{d\xi}{2\pi},
$$
i.e. the spectral density of $X$ is 
$$
F_X(\xi) = \frac{\sigma^2}{2\pi(\alpha^2+\xi^2)}.
$$

\paragraph*{(b)}

If $X$, $Y$ are independent then
\begin{align*}
C_{W}(t,s) & = \cos kt \, \cos ks \, C_X(t-s)  + \sin kt \, \sin ks \, C_Y(t-s) = \int e^{i\xi (t-s)}\cos k(t-s) \frac{\sigma^2}{\alpha^2+\xi^2}\frac{d\xi}{2\pi}
\\ & = \int e^{i\xi (t-s)}(e^{ik(t-s)}+e^{-ik(t-s)}) \frac{\sigma^2}{\alpha^2+\xi^2}\frac{d\xi}{4\pi} \\ & = \int e^{i\xi (t-s)} \frac{\sigma^2}{4\pi}\parq*{\frac{1}{\alpha^2+(\xi-k)^2}+\frac{1}{\alpha^2+(\xi+k)^2}}\,d\xi,
\end{align*}
i.e. the spectral density of $W$ is 
$$
F_W(\xi) = \frac{\sigma^2}{4\pi}\parq*{\frac{1}{\alpha^2+(\xi-k)^2}+\frac{1}{\alpha^2+(\xi+k)^2}}.
$$

\section*{Exercise 7}

\paragraph*{(a)}

Equation (a) means that
$$
X_t = \xi + \int_0^ts\,ds + 2\int_0^tdB_s = \xi + \frac{t^2}{2} + 2B_t.
$$
This implies $\expval X_t = t^2/2$.

\paragraph*{(b)}

If we multiply both sides of equation (b) by $e^{\cos t}$, we get
$$
d(e^{\cos t}X_t) = e^{\cos t}\,dX_t - (\sin t)e^{\cos t}X_t\,dt = e^{\cos t}\,dB_t,
$$
which gives
\begin{align*}
X_t & = \xi\,e^{(1-\cos t)} + e^{-\cos t}\int_0^te^{\cos s}\,dB_s  \\ & = \xi\,e^{(1-\cos t)} + e^{-\cos t}\int_0^td\parr*{e^{\cos s}B_s} -e^{-\cos t}\int_0^t(\sin s)B_s\,e^{\cos s}\,ds \\ & = \xi\,e^{(1-\cos t)} + B_t -e^{-\cos t}\int_0^t(\sin s)B_s\,e^{\cos s}\,ds.
\end{align*}
In particular $\expval X_t = 0$.

\paragraph*{(c)}

If we multiply both sides of equation (c) by $e^t$, we get
$$
d(e^tX_t) = e^t\,dX_t + e^tX_t\,dt = e^tdt + e^tdB_t,
$$
which gives
\begin{align*}
X_t & = \xi e^{-t} + e^{-t}\int_0^se^sds + e^{-t}\int_0^te^sdB_s
= \xi e^{-t} + 1 - e^{-t} + e^{-t}\int_0^td(e^sB_s) - e^{-t}\int_0^t e^sB_s\,ds \\ & = e^{-t}(\xi -1) + 1 + B_t - e^{-t}\int_0^t e^sB_s\,ds.
\end{align*}
In particular $\expval X_t = 1$.

\section*{Exercise 8}

\paragraph*{(a)}

By It\^o formula, it holds that
$$
dG_t = \frac{\alpha^2}{2}G_t\,dt - \alpha G_t\,dB_t + \frac{\alpha^2}{2}G_t\,dt = \alpha^2 G_t\,dt - \alpha G_t\,dB_t,
$$
and that
\begin{align*}
d(X_tG_t) & = X_t\,dG_t + G_t\,dX_t + dX_t\,dG_t \\ & = \alpha^2X_tG_t\,dt - \alpha X_tG_t\,dB_t + \frac{G_t}{X_t}\,dt + \alpha X_tG_t\,dB_t -\alpha^2X_tG_t\,dt = \frac{G_t}{X_t}\,dt.
\end{align*}

\paragraph*{(b)} By part (a), we have
\begin{equation}
dY_t = \frac{G_t}{X_t}\,dt \quad\Longrightarrow\quad Y_t = Y_0 + \int_0^t \frac{G_s}{X_s}\,ds = Y_0 + \int_0^t \frac{G_s^2}{Y_s}\,ds \quad\Longrightarrow\quad \frac{dY_t}{dt} = \frac{G_t^2}{Y_t}.
\label{ex:8}
\end{equation}

\paragraph*{(c)}
Equation (\ref{ex:8}) can be written as 
$$
\frac{d}{dt}\parr*{\frac{1}{2}Y_t^2} = G_t^2 \quad\Longrightarrow\quad Y_t^2 = Y_0^2 + 2\int_0^tG_s^2\,ds \quad\Longrightarrow\quad X_t^2 = \frac{X_0^2 + 2\int_0^tG_s^2\,ds}{G_t^2},
$$
which implies
$$
X_t = e^{\alpha B_t-\alpha^2t/2}\parq*{X_0^2+2\int_0^te^{\alpha^2s-2\alpha B_s}\,ds}^{1/2}.
$$

\section*{Exercise 9}

Suppose that $X_t$ satisfies the SDE 
\begin{equation}
dX_t = b(t,X_t)\,dt + \sigma(t,X_t)*dW_t
\label{ex9:*_int}
\end{equation}
and that this is equivalent to
\begin{equation} 
\label{ex9:int}
dX_t = \alpha(t,X_t)\,dt + \beta(t,X_t)\,dW_t
\end{equation}
for some functions $\alpha$, $\beta$. The terms in the Riemann sums
for (\ref{ex9:*_int}) is
\begin{align}
\Delta_j X & = b(t_j,X_j)\Delta_j t + \sigma(t_{j+1},X_{j+1})\Delta_j W \notag\\ & = b(t_j,X_j)\Delta_j t + \sigma(t_j,X_j)\Delta_jW+\partial_x\sigma(t_j,X_j)(\Delta_jX\Delta_jW) \notag\\ & = b(t_j,X_j)\Delta_j t + \sigma(t_j,X_j)\Delta_jW+\partial_x\sigma(t_j,X_j)
\cdot\beta(t_j,X_j)\Delta_jt \notag
\end{align}
where we used (\ref{ex9:int}) in the last equality. This means
\begin{equation}
dX_t = [b(t,X_t)+\partial_x\sigma(t,X_t)
\cdot\beta(t,X_t)]\,dt + \sigma(t,X_t)\,dW_t.
\label{ex9:int2}
\end{equation}
Comparing (\ref{ex9:int2}) with (\ref{ex9:int}) we conclude that it must be
$$
\beta(t,X_t) = \sigma(t,X_t) \qquad\text{and}\qquad \alpha(t,X_t) = b(t,X_t)+\partial_x\sigma(t,X_t)\cdot \sigma(t,X_t),
$$
and therefore $X$ must satisfy
$$
dX_t = [b(t,X_t)+\partial_x\sigma(t,X_t)\cdot \sigma(t,X_t)]\,dt + \sigma(t,X_t)\,dW_t.
$$

\end{document}